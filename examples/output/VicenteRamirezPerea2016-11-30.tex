% arara: pdflatex: { shell: yes}
% arara: pdflatex: { shell: yes}
% !TEX encoding = UTF-8
% !TEX program = pdflatex
% !TEX spellcheck = en_GB


\documentclass[english,a4paper,nologo]{europasscv}
 \usepackage[english]{babel}    



\usepackage{hyperref}
\usepackage{xcolor}
\definecolor{OliveGreen}{cmyk}{0.64,0,0.95,0.40}

\usepackage{cleveref}
\usepackage{booktabs}
\usepackage{graphicx}
\usepackage[space]{grffile}

% Used to force environment together. The environment must be established after the tex is generated, providing a way to control the appearance.
\newenvironment{absolutelynopagebreak}
{\par\nobreak\vfil\penalty0\vfilneg
	\vtop\bgroup}
{\par\xdef\tpd{\the\prevdepth}\egroup
	\prevdepth=\tpd}
% Cambiar Footnote to be shown on every page needs to uncomment next line and provide \fancyfoot after \begin{document}
\ecvfootnote{}  

% ============================================
% BEGIN_FOLD Datos personales

\ecvname{Vicente Ramírez Perea}
\ecvaddress{Watertorenstraat 25, 1560 Hoeilaart Belgium }
\ecvmobile{+32 475 51 49 13}

\ecvemail{vicente.ramirez.perea@gmail.com}
\ecvemail{vicente.ramirez.perea@outlook.com}

\ecvhomepage{http://vicente.ramirezperea.es }

\ecvim{Skype}{VicteAlias}
\ecvim{Google Hangouts}{vicente.ramirez.perea@gmail.com}

\ecvdateofbirth{30/05/1970}
\ecvnationality{Spanish}
\ecvgender{Male}
\ecvpicture[width=3.8cm,trim={0.4cm 0.9cm 0.5cm 0.4cm},clip]{../data/figures/vrpFormal.png}
\hypersetup{
	pdfauthor={Vicente Ramírez Perea}, 
	pdftitle={Curriculum vitae}, 
	%	pdfkeywords={xxxx}, %TODO: Keywords
	pdfsubject={},
	%    plainpages=false,
	pdfcreator={\LaTeX\ with package \flqq hyperref\frqq}
	pdfpagelabels,
	bookmarksopen=true,
	bookmarksnumbered=true,
	unicode=true,
	colorlinks=true,
	citecolor=blue,
	urlcolor=OliveGreen, % Requiere xcolors
	linkcolor=blue,
	anchorcolor=blue,
	filecolor=cyan,
	menucolor=red,
	runcolor=cyan,
	pdfborder={0 0 0}}%,
%	hyperfootnotes=true}
% END_FOLD

\begin{document}
	\fancyfoot[C]{Last digital version of this CV at \\ \url{https://drive.google.com/open?id=0B3n_KOODBfifNk1Gak1NbExmYms} }
	\begin{europasscv}
		\ecvpersonalinfo
				
		\ecvbigitem{Personal statement}{Multidisciplinar profesional ready to apply his knowledge and continue learning and improving.}


		\ecvsection{Work experience}
	
		\ecvtitle{June 2014 -- Present}{Personal projects}
		
			\ecvitem{\textit{Technologies: LaTeX, TexStudio, pdftk, Zotero, biblatex, Arara, Jenkins, git, bitbucket}
}{\textbf{PhD Thesis edition in Latex and management of bibliography}.
\url{https://drive.google.com/open?id=0B3n_KOODBfifYVI0UDFIaGNwRHc}.
More than 400 references managed, finally 267 used (\url{https://www.zotero.org/groups/thesis_elisa_vargas}).
}
		
			\ecvitem{\textit{Technologies: Python, yaml, jinja2 (templating html), html, Google Analytics, pandoc, markdown,
git, Bitbucket, Github}
}{\textbf{Another static website generator (ongoing project)}. Migrating an existing website to HTML5 induced the
development of the project. The aim is to create a script to be used by people with little knowledge of
programming for generating and deploying a website. They would only need to know basic HTML (or markdown).
Based in templates and directory structure. \url{git@github.com:victe/astawebsige.git}.
}
		
			\ecvitem{\textit{Technologies: LaTeX, Texstudio, Python, yaml, jinja2 (templating LaTeX), git, Gitlab, Github}
}{\textbf{Automatic CV generator -europass included- (ongoing project)}. Completing a europass has limitations,
even with the Office templates (MS, Libre or Open). Especially if you want to reuse your data in other format,
or in different languages. The solution adopted is templating with jinja2 (for LaTeX or HTML). Other
possibilities are using Sweave + pandoc or knitr. For the europass format there are already various document
classes in CTAN. The system stores data in yaml files, but it is easily trasformable to databases storage. I
selected yaml because it is easily modifiable in any place and it is not app dependant.
\url{git@github.com:victe/cv-templating.git}
}
		
			\ecvitem{\textit{Technologies: LaTeX, Texstudio, EjsS platform (formerly EJS), Java~8, git, Gitlab, Github}
}{\textbf{Simulations of experiments using Java}. Final project for my degree in physics (ongoing project,
Nov 2016 to Jun 2017). The objective is to build some virtual experiments of elastic pendulum as example of
chaotic systems using Easy Java Simulations platform (EjsS).
}
		
			\ecvitem{\textit{Technologies: detailed in project description}\newline
\includegraphics[]{figures/protoBar.jpg}
}{\textbf{Automatic testing system of some solid state physical properties of rod metals}. Final project for my
degree in physics (cancelled). Now I continue the work as a personal project. The goal is to build a low cost
experiment system for using it in classroom and remote controling it,  having implications on security
operation. \textit{Technologies: Programming:} Python, Flask, bokeh, PyCharm, pytest, conda, C++, Catch,
Eclipse, Git, BitBucket, Gitlab. \textit{Hardware and electronics:} Raspberry Pi, ADC, DAC, OAMP, IAMP, 1-WIRE,
Kelvin bridge, Qucs, MultiSIM, System Modeler. \textit{Measurement and calibration:} Oscilloscope, Source
current control, signal generator, balances and calibrated gauges, thermometers (solid-state digital and
analogue, platinum resistor, reference for calibration), piezoelectric sensors, voltmeters, ammeter.
\textit{Physical phenomena:} System Modeler, Wolfram Mathematica and IPython Notebooks. \textit{Analysis of
uncertainty:} theoretical with Wolfram Mathematica, simulations with System Modeler. Expression of uncertainty
following GUM guide and numerical derivation of uncertainties with an own module developed in R. \textit{Draw
up report and bibliography management:} LaTeX, Texstudio, Zotero, Biblatex. \textit{Graphical visualisation:}
ggplot, own function customisation in R. \textit{Management:} Trello, Google Keep.}
		
	


\end{europasscv}
\end{document}