% arara: pdflatex: { shell: yes}
% arara: pdflatex: { shell: yes}
% !TEX encoding = UTF-8
% !TEX program = pdflatex
% !TEX spellcheck = en_GB

((* set comma = joiner(", ") *))

\documentclass[((( doc_options )))]{europasscv}
((* for package in packages *)) ((* for key, value in package.items() *))\usepackage[(((value)))]{(((key)))}    ((* endfor *))
((* endfor *))
((( include_preamble|escape_tex )))

\usepackage{hyperref}
\usepackage{xcolor}
\definecolor{OliveGreen}{cmyk}{0.64,0,0.95,0.40}

\usepackage{cleveref}
\usepackage{booktabs}
\usepackage{graphicx}
\usepackage[space]{grffile}

% Used to force environment together. The environment must be established after the tex is generated, providing a way to control the appearance.
\newenvironment{absolutelynopagebreak}
{\par\nobreak\vfil\penalty0\vfilneg
	\vtop\bgroup}
{\par\xdef\tpd{\the\prevdepth}\egroup
	\prevdepth=\tpd}
% Cambiar Footnote to be shown on every page needs to uncomment next line and provide \fancyfoot after \begin{document}
\ecvfootnote{}  

% ============================================
% BEGIN_FOLD Datos personales

\ecvname{((( persinfo.fullname|escape_tex )))}
\ecvaddress{((( persinfo.address|escape_tex ))), ((( persinfo.postal_code ))) ((( persinfo.village|escape_tex ))) ( (((- persinfo.country|escape_tex -))) ) }
((* if persinfo.mobile != '' *))\ecvmobile{((( persinfo.mobile|escape_tex )))}((* endif *))
((* if persinfo.telephone != '' *))\ecvtelephone{((( persinfo.telephone|escape_tex )))}((* endif *))
((* for email in persinfo.emails *))\ecvemail{((( email|escape_tex )))}
((* endfor *))
\ecvhomepage{((* for web in persinfo.homepages *))((( web|escape_tex ))) ((* endfor *))}

((* for im in persinfo.ims *))((* for key, value in im.items() *))\ecvim{((( key|escape_tex )))}{((( value|escape_tex )))}((* endfor *))
((* endfor *))
\ecvdateofbirth{((( persinfo.dateofbirth|escape_tex )))}
\ecvnationality{((( persinfo.nationality|escape_tex )))}
\ecvgender{((( persinfo.gender|escape_tex )))}
\ecvpicture[width=3.8cm,trim={0.4cm 0.9cm 0.5cm 0.4cm},clip]{((( persinfo.image_file )))}
\hypersetup{
	pdfauthor={((( persinfo.fullname|escape_tex )))}, 
	pdftitle={((( pdftitle )))}, 
	pdfsubject={},
	%    plainpages=false,
	pdfcreator={\LaTeX\ with package \flqq hyperref\frqq}
	pdfpagelabels,
	bookmarksopen=true,
	bookmarksnumbered=true,
	unicode=true,
	colorlinks=true,
	citecolor=blue,
	urlcolor=OliveGreen, % Requiere xcolors
	linkcolor=blue,
	anchorcolor=blue,
	filecolor=cyan,
	menucolor=red,
	runcolor=cyan,
	pdfborder={0 0 0}}%,
%	hyperfootnotes=true}
% END_FOLD

\begin{document}
	\fancyfoot[C]{Last digital version of this CV at \\ \url{((( finalurl[language][1] )))} }
	\begin{europasscv}
% Personal info
		\ecvpersonalinfo
				
		\ecvbigitem{((( bigitem[0]|escape_tex )))}{((( bigitem[1]|escape_tex )))}

% Work experience
((* if profinfo is defined *))
		\ecvsection{Work experience}
	((* for section in profinfo *))
		\ecvtitle{((( section.period|escape_tex )))}{((( section.company|escape_tex )))}
		((* if section.infocompany is defined and section.infocompany != '' *))
			\ecvitem{}{((( section.infocompany )))}
		((* endif -*))
		((*- if section.certificate is defined and section.certificate != '' *))
			\ecvitem{}{ \href{((( section.certificate[1] )))}{((( section.certificate[0] )))}}
		((* endif *))
		((* if section.activities is defined and section.activities != '' *))
			\ecvitem{Main activities and responsibilities:}{
				\begin{ecvitemize}
				((* for i in section.activities *))\item ((( i )))
				((* endfor *))
				\end{ecvitemize}
				}
		((* endif *))
		((* if section.projects is defined *))
		\ecvitem{Main projects}{}
			((* for project in section.projects *))
				((* if project.name is defined *))
					\ecvitem{\textit{((( project.technologies )))}((* if project.figure is defined *))\newline
						\includegraphics[]{((( project.figure )))}((* endif *))}{-
					\textbf{((( project.name )))}. 
				
					((*- if project.time is defined and project.time != '' *)) ( (((- project.time -))) ). ((* endif *))
					((* if project.description is defined and project.description != '' *))
						((( project.description ))). 
					((* else *))((* if project.maindesc is defined *))
						((( project.maindesc ))) \newline
						\begin{ecvitemize}
							((* for l in project.detaildesc *))
							\item ((( l )))
							((* endfor *))
						\end{ecvitemize}((* endif *))
					((* endif *))
					((* if project.mainresponsabilities is defined and project.mainresponsabilities != '' *))\textit{Main responsabilities}: ((( project.mainresponsabilities ))).((* endif *))
					((* if project.urls is defined *)) 
						Refs.: 
						((* if project.urls|length > 1 *))
							((* for url in project.urls *))((( comma() )))\href{((( url[1] )))}{((( url[0] )))}((* endfor *))
						((* else *))
							((* for url in project.urls *))\href{((( url[1] )))}{((( url[0] )))}((* endfor *))
						((* endif *))
					((* endif *))
					}
				((* endif *))
			((* endfor *))
		((* endif *))
	((* endfor *))
((* endif *))

% Education and training
((* if acadinfo is defined *))
		\ecvsection{Education and training}
	((* for section in acadinfo *))
		\ecvtitlelevel{((( section.period )))}{((( section.title )))}{((( section.level )))}
		((* for center in section.centers *))
			\ecvitem{((( center.left )))}{((( center.right )))}
		((* endfor *))
	((* endfor *))
((* endif *))

% Personal skills
((* if persskills is defined *))
	\ecvsection{Personal skills}
	((* if persskills.langs is defined *))
		\ecvmothertongue{((( persskills.langs.mothertongue )))}
		\ecvlanguageheader
		((* for other in persskills.langs.others *))
			\ecvlanguage{((( other.lang )))}{((( other.understanding.listening )))}{((( other.understanding.reading )))}{((( other.speaking.interaction )))}{((( other.speaking.production )))}{((( other.writing )))}
			\ecvlanguagecertificate{((( other.certificate )))}
		((* endfor *))
		\ecvlanguagefooter
	((* endif *))
	((* if persskills.communicationskills is defined *))
		\ecvblueitem{Communication skills}{((( persskills.communicationskills )))}
	((* endif *))
	((* if persskills.managerialskills is defined *))
		\ecvblueitem{Organisational / managerial skills}{((( persskills.managerialskills )))}
	((* endif *))
	((* if persskills.jobrelatedskills is defined *))
		\ecvblueitem{Job-related skills}{((( persskills.jobrelatedskills )))}
	((* endif *))
	((* if persskills.digitalcompetence is defined *))
		\ecvdigitalcompetence{((( persskills.digitalcompetence.informationprocessing )))}{((( persskills.digitalcompetence.communication )))}{((( persskills.digitalcompetence.contentcreation )))}{((( persskills.digitalcompetence.safety )))}{((( persskills.digitalcompetence.problemsolving )))}
	((* endif *))
	((* if persskills.otherskills is defined *))
		\ecvblueitem{Other skills}{((( persskills.otherskills )))}
	((* endif *))
	((* if persskills.drivinglicence is defined *))
		\ecvblueitem{Driving licence}{((( persskills.drivinglicence )))}
	((* endif *))
((* endif *))

% Additional information
((* if additinfo is defined *))
	\ecvsection{Additional information}
	((* if additinfo.publications is defined *))
	\ecvblueitem{Publications}{((( additinfo.publications )))}
	((* endif *))
	((* if additinfo.other is defined *))
	\ecvblueitem{Other}{((( additinfo.other )))}
	((* endif *))
	((* if additinfo.computerskills is defined *))
	\ecvblueitem{Computer skills}{((( additinfo.computerskills )))}
	((* endif *))
((* endif *))

% Figures
\begin{figure}
	\caption{\textbf{Graphical representation of the evolution of my use of computer tools grouped by category}. Horizontal axis shows years.
		The intensity of the colours indicates time since the last use. More intense colour means more recent use and more faint indicates it was used longer time ago. The level expressed next to the name of the tools between parenthesis is a self-assessment considering the intensity of use during that period of time, the last time I used it, and my confidence with the tool. Graduation is $N\rightarrow Novice$, $I\rightarrow Intermediate$, $A\rightarrow Advanced$, and in some cases $L\rightarrow left~ out$. There is no \textit{Expert} level in my case, because I consider that implies continuous and almost exclusively use of such tool. In the graphs, some tools can be used in distinct Types and Categories with distinct self-assessment level of expertise.}
	\label{fig}
	\begin{tabular}{p{0.5\textwidth} p{0.5\textwidth}}
		\toprule
		\raisebox{-\totalheight}{
			\includegraphics[width=0.5\textwidth]{figures/Computing.pdf}} &
		\raisebox{-\totalheight}{
			\includegraphics[width=0.5\textwidth]{figures/Data.pdf}} \\
		
	
	\bottomrule
	
	\end{tabular}

\end{figure}
\begin{figure}
	\caption{\textbf{\cref{fig} cont.}. Horizontal axis shows years.
		The intensity of the colours indicates time since the last use. More intense colour means more recent use and more faint indicates it was used longer time ago. The level expressed next to the name of the tools between parenthesis is a self-assessment considering the intensity of use during that period of time, the last time I used it, and my confidence with the tool. Graduation is $N\rightarrow Novice$, $I\rightarrow Intermediate$, $A\rightarrow Advanced$, and in some cases $L\rightarrow left~ out$. There is no \textit{Expert} level in my case, because I consider that implies continuous and almost exclusively use of such tool. In the graphs, some tools can be used in distinct Types and Categories with distinct self-assessment level of expertise.}
	
\begin{tabular}{p{0.5\textwidth} p{0.5\textwidth}}
	\toprule
	
	\raisebox{-\totalheight}{
		\includegraphics[width=0.5\textwidth]{figures/Systems.pdf}} 	&
	
	\raisebox{-\totalheight}{
		\includegraphics[width=0.5\textwidth]{figures/Office tools.pdf}}  \\
	\raisebox{-\totalheight}{
		\includegraphics[width=0.5\textwidth]{figures/Professional.pdf}} &
	\raisebox{-\totalheight}{
		\includegraphics[width=0.5\textwidth]{figures/Management and communication.pdf}} 	
	
	\\
	
	\bottomrule
	
\end{tabular}
\end{figure}
\end{europasscv}
% TODO Link a Documento en otras lenguas
\end{document}
